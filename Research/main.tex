%! Author = thisi
%! Date = 12/05/2024

% Preamble
\documentclass[11pt]{article}

% Packages
\usepackage{amsmath}
\usepackage{geometry}
\geometry{
    a4paper,
    margin = 1in
}
\usepackage[english]{babel}
\usepackage{pgf}
\usepackage[utf8]{inputenc}
\usepackage{amssymb}
\usepackage{gensymb}
\usepackage{graphicx}
\usepackage{float}
\usepackage{authblk}
\usepackage{caption}
\usepackage{subcaption}
\usepackage{color}
\usepackage{tikzscale}
\usepackage{tikz,tikz-cd}
\usepackage{pgfplots}
\usepackage{braket}
\usepgfplotslibrary{external}
\tikzexternalize
\pgfplotsset{compat=1.18}
\usetikzlibrary{matrix, arrows}

\graphicspath{ {images/} }

\usepackage[nottoc]{tocbibind}

\newcommand{\dbar}{d\hspace*{-0.08em}\bar{}\hspace*{0.1em}}
\newcommand{\bO}{\mathcal{O}}

\setlength{\parindent}{0em}
\setlength{\parskip}{0em}

\title{Biome Generation Research}
\author{Joe Kent}

\begin{document}

    \maketitle

    \tableofcontents

    \section{Key Goals}\label{sec:key-goals}
    The focus of this project can be outlined as: Generating a diverse range of planet like objects with diverse biomes and features.
    This can be split into a number of sub goals: biomes, terrain, micro features, custom planet orientions/shapes, general climate structure and ability for customisation.
    \subsection{Biomes}\label{subsec:biomes}
    Earth has a wide range of biomes that are diverse and unique.
    However, these have rules for how they generate, as such we need to find a way to generate these in a logic way that allows for natural looking biome placement.
    Typically, there are three main factors that impact what a biome is like: Altitude, Water content (Precipitation, River flow, humidity) and Temperature.
    The temperature will mainly depend on the input from the planet's star and its orientation.
    The water content will depend on the wind, temperature, altitude and general planet water levels.
    See terrain (Section~\ref{subsec:terrain})to understand the approach fo altitude.
    Thus, the goal will be to generate these three factors in a realistic way to then impact the biome choice.
    The following is a list of biomes of interest and their general temperature, water content and altitude:
    \begin{table}
        \centering
        \caption{Table of biomes}
        \label{tab:biomes}
        \begin{tabular}{c | c c c c c c}
            Biome & Min Tem & Max Tem & Min Water & Max Water & Min Alt & Max Alt\\
            \hline
            Desert & &  &  &  &  & \\
            Savanna & &  &  &  &  & \\
            Temperate Forest & &  &  &  &  & \\
            Rainforest & &  &  &  &  & \\
            Deciduous Forest & &  &  &  &  & \\
            Tundra & &  &  &  &  & \\
            Ice Sheet & &  &  &  &  & \\
            Temperate & &  &  &  &  & \\
            Wetlands & &  &  &  &  & \\
            Swamps & &  &  &  &  & \\
            Highland Forest & &  &  &  &  & \\
            Shallow Seas & &  &  &  &  & \\
            Deep Oceans & &  &  &  &  & \\
            Coral Reefs & &  &  &  &  & \\
            Kelp Forest & &  &  &  &  & \\
            Snow Peaks & &  &  &  &  &
        \end{tabular}
    \end{table}
    However, even within these there is a wide diversity.
    For example, there is rocky desert, dune desert and mountainous desert.
    Each of these look very different, yet they are all classified as `desert'.
    Thus, it will be the aim of he terrain to generate this variety.

    \subsection{Terrain}\label{subsec:terrain}
    Within terrain there are a number of features which result in the diversity that we see.
    The following is a list of them.
    \begin{enumerate}
        \item Mountains
        \item Lakes
        \item Rivers
        \item Valleys
        \item Canyons
        \item Lone Mountians
        \item Glaciers
        \item Islands
        \item Continents
    \end{enumerate}
    Now to generate these features the base techniques that I plan to use are Techtonic Plates, Perlin Noise and Erosion.
    The Techtonic Plates should generate Continents, Islands, Mountain Ranges and Lone Mountains (or at least with some geothermal simulation).
    Perlin Noise will add some roughness to make it feel more like terrain.
    Erosion will be used to make Valleys, Canyons, Glaciers and Rivers.
    Erosion will likely be the most time consuming aspect of the project, both in implementation and compute time.
    \subsection{Micro Features}\label{subsec:micro-features}
    Within terrain there is a large diversity.
    For example, sometimes there is a lone mountain that was an old volcano, sometimes it is active.
    These features bring more life and diversity to a world but are not often large enough to be shown on a world map.
    Here is a list of these features.
    \begin{enumerate}
        \item Brine Lakes
        \item Caves
        \item Volcano
        \item Underwater Vent
        \item Braided Rivers
        \item Water fall
        \item Peaks
    \end{enumerate}
    The plan is to generate these things arround when the biomes are generated.
    Each feature will only be able to generate when specific conditions are met.
    \subsection{Custom Planet orientations and Shapes}\label{subsec:custom-planet-orientations-and-shapes}
    Spherical planets (or at least squished spheres) are what we commonly see in real life.
    However, there are more extreme examples of these or theoretically phyiscally posible shapes.
    I believe that these are the most interesting examples that one can obtain.
    Here is a list of the planned implemented features
    \begin{enumerate}
        \item spherical
        \item squished sphere
        \item tidally locked planet
        \item lonely planet
        \item binary suns
        \item ring world
        \item dyson world
        \item toriodal planet
        \item flat planet
        \item disk planet
    \end{enumerate}
    Most of these versions simply modify the temperature model or the slice where the techtonics or noise is generated from.
    However, some of these (like the toriodal planet) have modified gravity which would impact the generation of the planet.
    \subsection{Climate Structure}\label{subsec:climate-structure}
    The climate structure would mainly consist of wind and ocean currents.
    This would likely also impact the water content, temperature map and thus, biome map.
    The main impacts will likely be the wind cells, coriolis effect, pressure (from temperature and altitude).
    \subsection{User customisation}\label{subsec:user-customisation}
    This would mostly be things like custom biomes and custom initial heightmaps but could extend to adjusting biome placements or adding magical effects.
    To do the first two, storing biome types in csv and allowing the user to use a greyscale image to generate when generating the height map.

    \section{Research}
    General review~\cite{galin2019review}

    \subsection{Tectonics}
    One method of doing as seen in~\cite{cordonnier2016large}.
    \begin{enumerate}
        \item Generate a graph of points
        \item Compute flow between the points in the graph.
        \item Modify the trees if needed
        \item uplift and apply erosion
        \item Return to step 1 until convergence
    \end{enumerate}
    This method would result in mountains and rivers.
    However this requires an uplift step that is not outlined.
    Most online versions use Voronoi tessalation.
    Will likely use this and slicing a 3d version of the tessalation.
    Generate a loose grid of coordinates

    There are two types of plates, oceanic and continental plate, oceanic is denser.
    As well as this each plate has velocity.
    Plates that are next to each other have plate boundaries.
    These plate boundaries come in the following forms:
    Transform boundaries are boundaries that move parallel to the plate boundary.
    Not many features are at these.
    Colliding boundaries have a number of different results depending on what is colliding.
    First type is subduction which happens when oceanic crust meets continental crust.
    The oceanic crust sinks creating volcanos and the plate rising on the continental plates and a trench on the oceanic side .
    When two oceanic plates collide we get the same thing except the volcanos form islands and usually the older crust stays on top
    WHen two continental plates collide they form massive mountains.
    When two plates diverge they produce a rift zone which has volcanic activity in the center (it also forms dips in the land).
    There are also hot spots which could be anywhere.
    This hot spot does not move with the plates

    \subsection{Biomes}
    For Oceans the criteria is that the terrain is below sea level and temperature above -5
    If it is below -5 then it becomes ice sheet.
    Desert, extremely low rainfall (could make cold desert, warm desert and hot desert)
    Chaparral, low rainfall mainly in winter, dry summers
    Tundra, proportion of growing time (time above 0 degrees)
    Taiga, fifty fifty snow/dry warm/wet
    Savanna, lots of rainfall then droughts
    Temperate grasslands, moderate precipitation, cold winters and hot summers
    Tropical Rainforest, heavy precipitation, warm
    Temperate Rainforest, heavy precipitation, cool
    Temperate Deciduous Forest, distinct seasons and consistent precipitation
    Alpine, high altitudes
    Lakes or rivers, high water_level
    wetlands, medium water_level
    Intertidal zone, next to shoreline
    Neritic zone, sun can penetrate still
    Pelagic zone, deeper zone
    Benthis zone, deeper zone
    Abyssal zone, deepest zone
    Coral Reef, shallow parts of tropical ocean
    

    \section{Implementation}\label{sec:implementation}

    \subsection{Plan}\label{subsec:plan}
    Here is the vague plan, getting one of the steps done a week would be good (although some will likely take way less than that).
    \begin{itemize}
        \item Step 1: Research and Implement Techtonics
        \item Step 2: Implement Perlin Noise
        \item Check Point 1: Are we getting the general terrain shape that we want?
        \item Step 3: Research and Implement Temperature Map
        \item Step 4: Research and Implement Wind Map
        \item Step 5: Research and Implement Precipitation
        \item Check Point 2: Does the weather map look realistic?
        \item Step 6: Research and Implement Erosion
        \item Check Point 3: Are all the terrain features appearing?
        \item Step 7: Generate Biome Map
        \item Check Point 4: Does the biome map look realistic and good?
        \item Step 8: Generate Micro Features
        \item Check Point 5: Is the diversity and concentration of these features good?
        \item Step 9: Research and Implement Planet Variety
        \item Step 10: Implement Customisation
    \end{itemize}
    Also the algorithm should produce something that looks good, has a large diversity, can generate realistic looking planets and can generate a planet in 10minutes to an hour on my hardware.
    \subsection{Tectonics}\label{subsec:tectonics}
    Use a Voronoi Tessalation in 3d space to get the tectonic plate shapes.
    For each plate generate a plate type, velocity and weight (for the oceanic plates).
    Generate a grid of points with their heights (oceanic plates are lower) and assign each to their plates.
    Generate nodes with more detail near the plate edges.
    Now the simulation:
    generate the uplift
    Simulate node flows
    Run an erosion algorithm

    To generate the points we generate a set of random points in the domain $[-\pi/2, \pi/2]\otimes[0, 2*\pi]\subset \R^2$ before converting this to cartesian coordinates.
    We will then store the data in a quadtree to allow for points to quickly compute which is the closest
    \section{Test}\label{sec:test}
    \subsection{Test 1}
    Testing plate movement, am getting massive buildups at 0, $\pi$ and $2\pi$ not sure why
    \bibliographystyle{alpha}
    \bibliography{bibliograhpy}

\end{document}